% Created 2023-07-06 Thu 11:22
% Intended LaTeX compiler: pdflatex
\documentclass[11pt]{article}
\usepackage[utf8]{inputenc}
\usepackage[T1]{fontenc}
\usepackage{graphicx}
\usepackage{longtable}
\usepackage{wrapfig}
\usepackage{rotating}
\usepackage[normalem]{ulem}
\usepackage{amsmath}
\usepackage{amssymb}
\usepackage{capt-of}
\usepackage{hyperref}
\usepackage{minted}
\makeatletter
\newcommand{\citeprocitem}[2]{\hyper@linkstart{cite}{citeproc_bib_item_#1}#2\hyper@linkend}
\makeatother
\usepackage{amsmath}
\usepackage[margin=0.75in]{geometry}
\usepackage{parskip}
\usepackage{natbib}
\usepackage{dsfont}
\setcounter{secnumdepth}{2}
\author{Aaron Osgood-Zimmerman}
\date{\today}
\title{Spatial Statistics on Spatial Transcriptomics}
\hypersetup{
 pdfauthor={Aaron Osgood-Zimmerman},
 pdftitle={Spatial Statistics on Spatial Transcriptomics},
 pdfkeywords={},
 pdfsubject={Stats ideas, comments, and notes for spatial transcriptomics data and problems. Collab with MS Raredon},
 pdfcreator={Emacs 28.1 (Org mode 9.5.4)},
 pdflang={English}}
\begin{document}

\maketitle
\tableofcontents


\section{TODOS}
\label{sec:orgcd91fc9}
\subsection{Resources to eat [0/5]}
\label{sec:org3b21474}
\begin{itemize}
\item[{$\square$}]

\item[{$\square$}]

\item[{$\square$}]

\item[{$\square$}]

\item[{$\square$}]
\end{itemize}

\section{Useful resources}
\label{sec:org67cc984}
\subsection{Spatial Statistics}
\label{sec:orgb08dbcb}
\subsubsection*{\href{https://www.tandfonline.com/doi/full/10.1080/00401706.2018.1524791}{OSSEs} (\citeprocitem{1}{Ma et al. 2019})}
\label{sec:org5b3f7de}
\subsection{Spatial Transcriptomics}
\label{sec:org0a981ca}

\section{Big Ideas}
\label{sec:org86c5c02}

\subsection{Spatial Downscaling}
\label{sec:orgcf8db8a}

There are different possible topical applications, but the main
idea would be to use developed spatial statistics models that
acknowledge and account for coarser resolution(s) of the data
observations to model a finer resolution latent (unobserved)
truth.

Examples include coarse satellite measurements used to
model finer resolution weather/climate, estimating county-level
properties from state-level measurements, etc.

This has very a very clear, understandable, and interpretable
analogy to spatial transcriptomic data where the measurements are
coarser than the single cells that we would like to directly
model.

\subsubsection*{Spatial Downscaling Methods/Models}
\label{sec:org135beaf}
\paragraph*{Observing system simulation experiments (OSSE)}
\label{sec:org19ea205}

\printbibliography[../spatial-transcriptome-lib.bib]
\end{document}